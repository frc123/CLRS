\input{../tex_header}

\title{Chapter 15 Solusion}
\date{11/3/2021}

\begin{document}
\maketitle

\section*{15.1}

\subsection*{15.1-1}

\begin{proof}
    We prove by substitution method.
    For $n=0$, $T(0)=2^0=1$.
    For $n>0$,

    $T(n)=1+\sum\limits_{j=0}^{n-1}T(j)
    =1+\sum\limits_{j=0}^{n-1}2^j=1+(2^n-1)=2^n$
\end{proof}

\subsection*{15.1-2}

Consider the following case:

\begin{tabular}{c||c|c|c}
    length $i$      & 1 & 2 & 3 \\
    price $p_i$     & 1 & 6 & 8 \\
    density $p_i/i$ & 1 & 3 & 2.67
\end{tabular}

\noindent
If we use ``greedy'' strategy, our solution will be ``$2$ $1$'',
and the total price will be $7$.
However, the optimal way is ``$3$'', and the total price is $8$.

\subsection*{15.1-3}

\begin{minted}[xleftmargin=20pt,linenos]{cpp}
    /**
    * running time: O(n^2)
    * p: table of prices (index start from 0)
    * n: length of rod
    * c: cost of each cut
    * return maximum revenue
    */
    int BottomUpCutRodWithCost(const std::vector<int>& p, int n, int c)
    {
        int *r, q, i, j;
        r = new int[n + 1];
        r[0] = 0;
        for (j = 1; j <= n; ++j)
        {
            q = p[j - 1];
            for (i = 0; i < j - 1; ++i)
                q = std::max(q, p[i] + r[j - i - 1] - c);
            r[j] = q;
        }
        delete[] r;
        return q;
    }
\end{minted}

\subsection*{15.1-4}

\begin{minted}[xleftmargin=20pt,linenos]{cpp}
    /**
    * p: table of prices (index start from 0)
    * n: length of rod
    * r: table of maximum revenue (index start from 1)
    * s: table of optimal size i of the first piece to cut off (index start from 1)
    * return maximum revenue
    */
    int ExtendedMemoizedCutRodAux(const std::vector<int>& p, int n, int *r, int *s)
    {
        int q, i, reminder_r;
        if (r[n] >= 0) return r[n];
        q = INT_MIN;
        for (i = 0; i < n; ++i)
        {
            reminder_r = ExtendedMemoizedCutRodAux(p, n - i - 1, r, s);
            if (q < p[i] + reminder_r)
            {
                q = p[i] + reminder_r;
                s[n] = i + 1;
            }
        }
        r[n] = q;
        return q;
    }

    /**
    * running time: O(n^2)
    * p: table of prices (index start from 0)
    * n: length of rod
    * return (r, s)
    * r: table of maximum revenue (index start from 1)
    * s: table of optimal size i of the first piece to cut off (index start from 1)
    * caller is responsible to deallocate return value r and s
    */
    std::pair<int*, int*> ExtendedMemoizedCutRod(const std::vector<int>& p, int n)
    {
        int *r, *s, i;
        r = new int[n + 1];
        s = new int[n + 1];
        r[0] = 0;
        s[0] = 0;
        for (i = 1; i <= n; ++i) r[i] = INT_MIN;
        ExtendedMemoizedCutRodAux(p, n, r, s);
        return std::make_pair(r, s);
    }
\end{minted}

\subsection*{15.1-5}

\begin{minted}[xleftmargin=20pt,linenos]{cpp}
    /**
    * running time: O(n)
    * n: n-th fibonacci number (must greater than 0)
    */
    int FibonacciNumber(int n)
    {
        int *f, i, result;
        f = new int[n + 1];
        f[0] = 0;
        f[1] = 1;
        for (i = 2; i <= n; ++i)
            f[i] = f[i - 1] + f[i - 2];
        result = f[n];
        delete[] f;
        return result;
    }
\end{minted}

\section*{15.2}

\subsection*{15.2-1}

Optimal parenthesization: $((1,2),((3,4),(5,6)))$

Minimum cost: $2010$

\subsection*{15.2-2}

\begin{minted}[xleftmargin=20pt,linenos]{cpp}
    /**
    * s: table (2d) storing index of k achieved the optimal cost 
    *      (index start by 1)
    * caller is responisble to deallocate the return value
    */
    Matrix* MatrixChainMultiply
        (const std::vector<Matrix*>& matrices, const Table* s, int i, int j)
    {
        Matrix *matrix_a, *matrix_b, *matrix_c;
        if (i == j)
        {
            return matrices[i - 1];
        }
        matrix_a = MatrixChainMultiply(matrices, s, i, (*s)[i][j]);
        matrix_b = MatrixChainMultiply(matrices, s, (*s)[i][j] + 1, j);
        matrix_c = MatrixMultiply(matrix_a, matrix_b);
        if (i != (*s)[i][j]) delete matrix_a;
        if ((*s)[i][j] + 1 != j) delete matrix_b;
        return matrix_c;
    }
\end{minted}

\subsection*{15.2-3}

\begin{proof}
    \noindent
    We prove by substitution method.
    For $n=1$, $P(1)=1 \geq 2^k$ for $k\leq 0$.
    For $n \geq 2$,

    $P(n) = \sum\limits_{k=1}^{n-1}P(k)P(n-k)
    \geq \sum\limits_{k=1}^{n-1}(c \cdot 2^k)(c \cdot 2^{n-k})
    = \sum\limits_{k=1}^{n-1}(c^2 \cdot 2^n)
    = (n-1)(c^2 \cdot 2^n)
    \geq c^2 \cdot 2^n$

    \noindent
    for some constant $c$.
\end{proof}

\subsection*{15.2-4}
\noindent
For all vertices $v_{i,j}$ in the graph, 
it contains edge $(v_{i,j}, v_{i,k})$ and $(v_{i,j}, v_{k+1,j})$
for all $i \leq k < j$.

\noindent
Vertices: 

$\tbinom{n}{2} + n = \frac{n(n-1)}{2} + n = \frac{n(n+1)}{2}$

\noindent
Edges: 

$\sum\limits_{i=1}^{n}\sum\limits_{j=i}^{n}(j-i)
= \sum\limits_{i=1}^{n}(\sum\limits_{j=i}^{n}(j)-\sum\limits_{j=i}^{n}(i))
= \sum\limits_{i=1}^{n}(\sum\limits_{j=1}^{n}(j)-\sum\limits_{j=1}^{i-1}(j)-(n-i+1)i)$

$= \sum\limits_{i=1}^{n}(\frac{n(n+1)}{2}-\frac{(i-1)i}{2}-(n-i+1)i)
= \frac{n^2(n+1)}{2}-\sum\limits_{i=1}^{n}(\frac{i-i^2+2ni}{2})$

$= \frac{n^2(n+1)}{2}+\frac{1}{2}\sum\limits_{i=1}^{n}(i^2)-\frac{1}{2}(1+2n)\sum\limits_{i=1}^{n}(i)
= \frac{n^2(n+1)}{2}+\frac{n(n+1)(2n+1)}{12}-\frac{n(n+1)(2n+1)}{4}$

$= \frac{n^2(n+1)}{2}-\frac{n(n+1)(2n+1)}{6}
= \frac{(n-1)n(n+1)}{6}$

\subsection*{15.2-5}

\begin{proof}
    \noindent
    Notice that 
    $\sum\limits_{i=1}^{n}\sum\limits_{j=i}^{n}R(i,j)$ 
    is equal to the total times of any entries are referenced 
    during the entire call of $\proc{Matrix-Chain-Order}$.
    In other words, it is equal to twice the times of line $10$ 
    was executed during the entire call.

    \noindent
    Hence, we have

    $\sum\limits_{i=1}^{n}\sum\limits_{j=i}^{n}R(i,j)
    = \sum\limits_{l=2}^{n}\sum\limits_{i=1}^{n-l+1}\sum\limits_{k=i}^{i+l-2}2
    = 2\sum\limits_{l=2}^{n}(n-l+1)(l-1)
    = 2((n+2)\sum\limits_{l=2}^{n}l-\sum\limits_{l=2}^{n}l^2-(n-1)(n+1))$
    
    $= 2((n+2)(\frac{n(n+1)}{2}-1)-(\frac{n(n+1)(2n+1)}{6}-1))
    = \frac{n^3-n}{3}$
\end{proof}

\subsection*{15.2-6}

\begin{proof}
    \noindent
    We prove by induction.
    Let $P(n)$ be the claim:
    A full parenthesization of an $n$-element expression 
    has exactly $n-1$ pairs of parentheses.

    \noindent
    (\textit{Base Case})
    A $2$-element full parenthesization $(A_1,A_2)$ 
    has only one pair of parentheses clearly.
    Hence, we have proved $P(2)$ is true.

    \noindent
    (\textit{Induction Step})
    Suppose that $P(n)$ is true.
    Let $C$ be a sequence with $n+1$ elements: 
    $A_1A_2...A_nA_{n+1}$.
    Delete one arbitrary element from $C$, 
    we have a sequence with $n$ elements.
    By induction hypothesis, 
    $C$ ($n$-element) has exactly $n-1$ pairs of parentheses now.
    Add the deleted element back,
    we can add one pair of parentheses to surround the deleted element
    and one of the element's neighbor element 
    or one of the element's neighbor parenthesization.
    This says, $C$ ($n+1$-element) has exactly 
    $n$ pairs of parentheses now.
    We have proved $P(n+1)$ is true.
\end{proof}

\section*{15.3}

\subsection*{15.3-1}

\noindent
$\proc{Recursive-Matrix-Chain}$ is a more efficient way.

\begin{proof}
    \noindent
    By recurrence (15.6) in section 15.2, 
    there are $P(n)$ alternative parenthesizations of a sequence of matrices
    where

    $P(n)=
    \begin{cases}
        1 & \text{if $n = 1$.} \\
        \sum\limits_{k=1}^{n-1}P(k)(n-k) & \text{if $n \geq 2$.}
    \end{cases}
    \indent\indent\indent\indent\indent
    (15.6)$

    \noindent
    By problem 12-4, we proved that $P(n) = \Omega(4^n/n^{3/2})$.
    This says enumerating takes $\Omega(4^n/n^{3/2})$ time.

    \noindent
    In order to prove that 
    $\proc{Recursive-Matrix-Chain}$ is a more efficient than enumerating,
    we just need to prove that 
    $\proc{Recursive-Matrix-Chain}$ takes $o(4^n/n^{3/2})$ time.

    \noindent
    By recurrence (15.7) in section 15.2, 
    $\proc{Recursive-Matrix-Chain}$ takes $T(n)$ times
    where 

    $T(n)=
    \begin{cases}
        1 & \text{if $n = 1$.} \\
        1 + \sum\limits_{k=1}^{n-1}(T(k) + T(n-k) + 1) & \text{if $n \geq 2$.}
    \end{cases}$

    $T(n)=2\sum\limits_{i=1}^{n-1}T(i)+n$

    \noindent
    This says we want to prove that $T(n) = o(4^n/n^{3/2})$.
    Since $\lim\limits_{n\rightarrow\infty}\frac{4^n/n^{3/2}}{3.5^n}=\infty$,
    we just need to prove that $T(n) = O(3.5^n)$.
    ($T(n) = O(3^n)$ is false, so we try $T(n) = O(3.5^n)$.)

    \noindent
    We claim that $T(n) = O(3.5^n)$ and prove this by subsitution method.
    Let $c$ be some constant. Assume $T(n) \leq c \cdot 3.5^n$.

    \begin{equation*}
    \begin{split}
        T(n) &= 2 \sum\limits_{i=1}^{n-1} T(i) + n \\
        &\leq 2 \sum\limits_{i=1}^{n-1} (c \cdot 3.5^i) + n \\
        &= 2c (\frac{3.5^n - 1}{3.5 - 1} - 1) + n \\
        &= 2c (\frac{3.5^n - 3.5}{2.5}) + n \\
        &= 0.8c \cdot 3.5^n - 2.8c + n
    \end{split}
    \end{equation*}

    \noindent
    Let $c = 1$.
    We have 
    $0.8c \cdot 3.5^n - 2.8c + n \leq c \cdot 3.5^n$.
    We have proved $T(n) = O(3.5^n)$.
    Hence, $T(n) = o(4^n/n^{3/2})$.
\end{proof}

\subsection*{15.3-2}

    \Tree [.1..16 
        [.1..8
            [.1..4 [.1..2 1 2 ] [.3..4 3 4 ]
            ] [.5..8 [.5..6 5 6 ] [.7..8 7 8 ]
            ]
        ] [.9..16
            [.9..12 [.9..10 9 10 ] [.11..12 11 12 ]
            ] [.13..16 [.13..14 13 14 ] [.15..16 15 16 ]
            ]
        ]
    ]

    \noindent
    We notice that there is no overlapping subproblem,
    so memoization does not help to spped up the algorithm.

\subsection*{15.3-3}

\noindent
Yes.

\begin{proof}
    \noindent
    Let $A_{i...j}$ denotes sequece of matrices $A_i A_{i+1} ... A_j$.

    \noindent
    The subproblems in maximize multiplication are independent. 
    (for more information, refer to page 383)
    An optimal parenthesization of $A_{i...j}$
    that splits the product between $A_k$ and $A_{k+1}$
    contains within it optimal solutions
    to the problems of parenthesizing
    $A_{i...k}$ and $A_{k+1...j}$.

    \noindent
    Given parenthesization $P_{ij}$ maximize 
    the number of scalar multiplications to $A_{i...j}$,
    and $P_{ij}$ splits the product between $A_k$ and $A_{k+1}$.
    Making the choice (splits the product between $A_k$ and $A_{k+1}$)
    leaves subproblems 
    $A_{i...k}$ and $A_{k+1...j}$ to solve.
    Let $P_{ik}$ and $P_{k+1,j}$ be the parenthesization on
    $A_{i...k}$ and $A_{k+1...j}$ respectively.
    We want to prove that $P_{ik}$ and $P_{k+1,j}$ maximize 
    the number of scalar multiplications to
    $A_{i...k}$ and $A_{k+1...j}$ by contradiction.
    Suppose that 
    $P_{ik}$ and $P_{k+1,j}$ does not maximize 
    the number of scalar multiplications to 
    $A_{i...k}$ and $A_{k+1...j}$.
    Let $Q_{ik}$ and $Q_{k+1,j}$ be the optimal parenthesization 
    (maximize number of multiplications) to 
    $A_{i...k}$ and $A_{k+1...j}$.
    Then, by ``cutting out'' $P_{ik}$ and $P_{k+1,j}$
    and ``pasting in'' $Q_{ik}$ and $Q_{k+1,j}$, 
    we get a better solution (more number of multiplications) 
    to the original problem than $P_{ij}$.
    This contradicts to parenthesization $P_{ij}$ maximize 
    the number of scalar multiplications.
\end{proof}

\subsection*{15.3-4}

\noindent
Consider the following $p$'s:

\begin{tabular}{c|c|c|c}
    $p_0$ & $p_1$ & $p_2$ & $p_3$ \\
    \hline
    1 & 10 & 20 & 100
\end{tabular}

\noindent
By the approach of greedy algorithm,
we choose $k=1$ for $[i,j]=[1,3]$ since
$p_0p_1p_3=1000$ and $p_0p_2p_3=2000$.
Hence, the solution of greedy algorithm is $(A_1(A_2A_3))$.

\noindent
However, $((A_1A_2)A_3)$ ($k=2$) is the optimal solution,
which takes $p_0p_1p_2 + p_0p_2p_3 = 200 + 2000 = 2200$ multiplications.
Greedy solution $(A_1(A_2A_3))$ ($k=1$) takes 
$p_1p_2p_3 + p_0p_1p_3 = 20000 + 1000 = 21000$ multiplications.

\noindent
How can we find the counterexample?
We start to try to find a counterexample 
in a sequence with $3$ matrices.
This says $[i,j]=[1,3]$,
and there are two choices for $k$: $1$ or $2$.

\noindent
The algorithm perform $m[1,3]$ times multiplication.

$m[1,3]=
\begin{cases}
    m[1,1]+m[2,3]+p_0p_1p_3 & k=1 \\
    m[1,2]+m[3,3]+p_0p_2p_3 & k=2
\end{cases}
=
\begin{cases}
    p_1p_2p_3+p_0p_1p_3 & k=1 \\
    p_0p_1p_2+p_0p_2p_3 & k=2
\end{cases}$

\noindent
We try to make the greedy algorithm choose $k=1$.
This says we want $p_0p_1p_3<p_0p_2p_3 \Longleftrightarrow p_1<p_2$.
In order to make the greedy approach ($k=1$) yields a suboptimal solution,
we want $k=2$ to be the optimal approach.
This says we want $p_1p_2p_3+p_0p_1p_3 > p_0p_1p_2+p_0p_2p_3$.

\noindent
Hence, our goal is to find $p_0,p_1,p_2,p_3$ such that 
$p_1<p_2$ and $p_1p_2p_3+p_0p_1p_3 > p_0p_1p_2+p_0p_2p_3$.
We try to let $p_1 = 10$ and $p_2 = 20$.
By a sloppy way, we can try to let $p_3$ much larger than $p_0$
since $p_3$ appears twice and $p_0$ appears once on the LHS,
and $p_0$ appears twice and $p_3$ appears once on the RHS.
We try to let $p_0 = 1$ and $p_3 = 100$.
After testing, we find that this is a good counterexample.

\subsection*{15.3-5}

\noindent
If we have limit $l_i$ on the number of pieces of length $i$ 
that we are allowed to produce,
We can not find the optimal subproblems indenpendently.
We show the optimal-substructure property does not hold
by providing a counterexample. 
(Recall optimal substructure on page 374)

\noindent
Consider the following case:

\begin{tabular}{c||c|c|c}
    $i$     & 1 & 2 & 3 \\
    \hline
    $p_i$   & 5 & 8 & 9 \\
    $l_ii$  & 2 & 2 & 1
\end{tabular}

\noindent
The optimal solution of cutting the rod where $i=3$ is 
lengths $1$ and $2$ with price $5+8=13$.
However, the optimal solution of cutting the rod where $i=2$ is
lengths $1$ and $1$ with price $5+5=10$.
We have showed that there is a way of cutting that 
does not contain in the optimal solution to the original problem
but does contain in the optimal solution to the subproblem,
which violates the optimal-substructure property.

\subsection*{15.3-6}

\noindent
Note: For this question, we assume that $r_{ij}r_{ji}=1$ 
for any $1 \leq i,j \leq n$.

\begin{claim}
    If $c_k = 0$ for all $k = 1,2...,n$, 
    then the problem of finding the best sequence of exchanges 
    from currency $1$ to currency $n$ exhibits optimal substructure.
\end{claim}

\begin{proof}
    Let the sequence of currencies $k_1,k_2,k_3,...,k_{n-1},k_n$ 
    be the best sequence
    to exchange from currency from $k_1$ to currency $k_n$,
    which means $r_{k_1k_2}r_{k_2k_3}...r_{k_{n-1}k_n}$ is maximized.
    We show that the sequence of currencies 
    $k_i,k_{i+1},...,k_{j-1},k_{j}$ is the best sequence to exchange
    from currency from $k_i$ to currency $k_j$ by contradiction.
    Assume $k_i,q_{i+1},...,q_{j-1},k_{j}$ is the best sequence to exchange
    from currency from $k_i$ to currency $k_j$.
    By using the ``cut-and-paste'' technique to replace 
    $k_i,k_{i+1},...,k_{j-1},k_{j}$ with
    $k_i,q_{i+1},...,q_{j-1},k_{j}$,
    we get a better sequence of currencies to 
    exchange from currency from $k_1$ to currency $k_n$
    (e.g. $k_1,k_2,k_3,...k_i,q_{i+1},...,q_{j-1},k_{j},...,k_{n-1},k_n$),
    which contradicts to $k_1,k_2,k_3,...,k_{n-1},k_n$ is the best sequence.
\end{proof}

\noindent
Now, we show that if $c_k$ are arbitrary values, 
then the problem of finding the best sequence of exchanges 
from currency $1$ to currency $n$ does not
necessarily exhibit optimal substructure.

\noindent
Consider the following case:

\noindent
We want to exchange from currency from $k_1$ to currency $k_4$.

\begin{tabular}{c|c|c|c|c|c}
    $r_{k_1k_2}$ & $r_{k_2k_3}$ & $r_{k_3k_4}$
    & $r_{k_1k_3}$ & $r_{k_2k_4}$ & $r_{k_1k_4}$ \\
    \hline
    6 & 2 & 5 & 2 & 8 & 10
\end{tabular}
\indent\indent\indent
\begin{tabular}{c|c|c}
    $c_1$ & $c_2$ & $c_3$ \\
    \hline
    4 & 5 & 20
\end{tabular}

\noindent
We try to find the optimal sequence from $k_1$ to $k_4$ 
by list all possible sequences:

$k_1k_4$: $r_{k_1k_4} - c_1 = 10 - 4 = 6$

$k_1k_2k_4$: $r_{k_1k_2}r_{k_2k_4} - c_2 = 6 \cdot 8 - 5 = 43$

$k_1k_3k_4$: $r_{k_1k_3}r_{k_3k_4} - c_2 = 2 \cdot 5 - 5 = 5$

$k_1k_2k_3k_4$: $r_{k_1k_2}r_{k_2k_3}r_{k_3k_4} - c_3 
= 6 \cdot 2 \cdot 5 - 20 = 40$

$k_1k_3k_2k_4$: $r_{k_1k_3}r_{k_3k_2}r_{k_2k_4} - c_3 
= 2 \cdot \frac{1}{2} \cdot 8 - 20 = -12$

\noindent
The optimal sequence from $k_1$ to $k_4$ is $k_1k_2k_4$.

\noindent
Now, we try to show that $k_2k_4$ is not the optimal sequence 
from $k_2$ to $k_4$ 
by list all possible sequences:

$k_2k_4$: $r_{k_2k_4} - c_1 = 8 - 4 = 4$

$k_2k_3k_4$: $r_{k_2k_3}r_{k_3k_4} - c_2 = 2 \cdot 5 - 5 = 5$

$k_2k_1k_4$: (unnecessary)

$k_2k_1k_3k_4$: (unnecessary)

$k_2k_3k_1k_4$: (unnecessary)

\noindent
It is unnecessary to solve results for 
$k_2k_1k_4$, $k_2k_1k_3k_4$, and $k_2k_3k_1k_4$ 
since we already find that 
$k_2k_3k_4$ is a better sequence than $k_4$ is $k_2k_4$.

\section*{15.4}

\subsection*{15.4-1}

$\langle 1, 0, 0, 1, 1, 0 \rangle$

(the solution is not unique)

\subsection*{15.4-2}

\begin{minted}[xleftmargin=20pt,linenos]{cpp}
    template <typename T>
    std::list<T> OutputLCS(LengthTable length_table, 
        const std::vector<T>& sequence_x, const std::vector<T>& sequence_y)
    {
        size_t it_x, it_y;
        std::list<T> common_sequence;
        it_x = sequence_x.size();
        it_y = sequence_y.size();
        while (it_x > 0 && it_y > 0)
        {
            const T& element_x = sequence_x[it_x - 1];
            const T& element_y = sequence_y[it_y - 1];
            if (element_x == element_y)
            {
                common_sequence.push_front(element_y);
                --it_x;
                --it_y;
            }
            else if (length_table[it_x - 1][it_y] >= length_table[it_x][it_y - 1])
            {
                --it_x;
            }
            else
            {
                --it_y;
            }
        }
        return common_sequence;
    }
\end{minted}

\subsection*{15.4-3}

\begin{minted}[xleftmargin=20pt,linenos]{cpp}
    template <typename T>
    int RecursiveLCSLengthMemoized
        (const std::vector<T>& sequence_x, const std::vector<T>& sequence_y, 
        size_t it_x, size_t it_y,
        LengthTable& length_table, PointerTable& pointer_table) 
    {
        int sequence_length_x_prev, sequence_length_y_prev;
        if (length_table[it_x][it_y] >= 0)
            return length_table[it_x][it_y];
        if (sequence_x[it_x - 1] == sequence_y[it_y - 1])
        {
            length_table[it_x][it_y] = RecursiveLCSLengthMemoized(sequence_x, sequence_y, 
                it_x - 1, it_y - 1, length_table, pointer_table) + 1;
            pointer_table[it_x][it_y] = PointerType::X_Y_PREV;
        }
        else
        {
            sequence_length_x_prev = RecursiveLCSLengthMemoized(sequence_x, sequence_y, 
                it_x - 1, it_y, length_table, pointer_table);
            sequence_length_y_prev = RecursiveLCSLengthMemoized(sequence_x, sequence_y, 
                it_x, it_y - 1, length_table, pointer_table);
            if (sequence_length_x_prev >= sequence_length_y_prev) 
            {
                length_table[it_x][it_y] = sequence_length_x_prev;
                pointer_table[it_x][it_y] = PointerType::X_PREV;
            } 
            else 
            {
                length_table[it_x][it_y] = sequence_length_y_prev;
                pointer_table[it_x][it_y] = PointerType::Y_PREV;
            }
        }
        return length_table[it_x][it_y];
    }

    template <typename T>
    std::pair<LengthTable, PointerTable> LCSLengthMemoized
        (const std::vector<T>& sequence_x, const std::vector<T>& sequence_y) 
    {
        size_t size_x, size_y, it_x, it_y, 
            sequence_length_x_prev, sequence_length_y_prev;
        size_x = sequence_x.size();
        size_y = sequence_y.size();
        // note this line init all elements in length_table with -1
        LengthTable length_table(size_x + 1, LengthTableRow(size_y + 1, -1));
        PointerTable pointer_table(size_x + 1, PointerTableRow(size_y + 1, NIL));
        for (it_x = 0; it_x <= size_x; ++it_x)
            length_table[it_x][0] = 0;
        for (it_y = 1; it_y <= size_y; ++it_y)
            length_table[0][it_y] = 0;
        RecursiveLCSLengthMemoized(sequence_x, sequence_y, size_x, size_y, 
            length_table, pointer_table);
        return std::make_pair(std::move(length_table), std::move(pointer_table));
    }
\end{minted}

\subsection*{15.4-4}

Solution for space of $2 \cdot \func{min}(m,n) + O(1)$:

\begin{minted}[xleftmargin=20pt,linenos]{cpp}
    template <typename T>
    std::pair<int, PointerTable> LCSLengthByTwoRowLengthTable
        (const std::vector<T>& sequence_x, const std::vector<T>& sequence_y) 
    {
        size_t size_x, size_y, it_x, it_y;
        int sequence_length_x_prev, sequence_length_y_prev;
        size_x = sequence_x.size();
        size_y = sequence_y.size();
        if (size_x < size_y)
            return LCSLengthByTwoRowLengthTable(sequence_y, sequence_x);
        LengthTableRow length_table_row_1(size_y + 1), length_table_row_2(size_y + 1);
        PointerTable pointer_table(size_x + 1, PointerTableRow(size_y + 1));
        for (it_y = 0; it_y <= size_y; ++it_y)
            length_table_row_1[it_y] = 0;
        for (it_x = 1; it_x <= size_x; ++it_x) 
        {
            for (it_y = 1; it_y <= size_y; ++it_y) 
            {
                if (sequence_x[it_x - 1] == sequence_y[it_y - 1]) 
                {
                    length_table_row_2[it_y] = 
                        length_table_row_1[it_y - 1] + 1;
                    pointer_table[it_x][it_y] = PointerType::X_Y_PREV;
                } 
                else 
                {
                    sequence_length_x_prev = length_table_row_1[it_y];
                    sequence_length_y_prev = length_table_row_2[it_y - 1];
                    if (sequence_length_x_prev >= sequence_length_y_prev) 
                    {
                        length_table_row_2[it_y] = sequence_length_x_prev;
                        pointer_table[it_x][it_y] = PointerType::X_PREV;
                    } 
                    else 
                    {
                        length_table_row_2[it_y] = sequence_length_y_prev;
                        pointer_table[it_x][it_y] = PointerType::Y_PREV;
                    }
                }
            }
            length_table_row_1.swap(length_table_row_2);
        }
        return std::make_pair(length_table_row_2[size_y], std::move(pointer_table));
    }
\end{minted}

Solution for space of $\func{min}(m,n) + O(1)$:

\begin{minted}[xleftmargin=20pt,linenos]{cpp}
    template <typename T>
    std::pair<int, PointerTable> LCSLengthByOneRowLengthTable
        (const std::vector<T>& sequence_x, const std::vector<T>& sequence_y) 
    {
        size_t size_x, size_y, it_x, it_y;
        int sequence_length_x_y_prev, sequence_length_x_y_now;
        size_x = sequence_x.size();
        size_y = sequence_y.size();
        if (size_x < size_y)
            return LCSLengthByOneRowLengthTable(sequence_y, sequence_x);
        LengthTableRow length_table_row(size_y + 1);
        PointerTable pointer_table(size_x + 1, PointerTableRow(size_y + 1));
        for (it_y = 0; it_y <= size_y; ++it_y)
            length_table_row[it_y] = 0;
        for (it_x = 1; it_x <= size_x; ++it_x) 
        {
            sequence_length_x_y_prev = 0;
            for (it_y = 1; it_y <= size_y; ++it_y) 
            {
                sequence_length_x_y_now = length_table_row[it_y];
                if (sequence_x[it_x - 1] == sequence_y[it_y - 1]) 
                {
                    length_table_row[it_y] = 
                        sequence_length_x_y_prev + 1;
                    pointer_table[it_x][it_y] = PointerType::X_Y_PREV;
                } 
                else 
                {
                    if (length_table_row[it_y] >= length_table_row[it_y - 1]) 
                    {
                        // length_table_row[it_y] = length_table_row[it_y];
                        pointer_table[it_x][it_y] = PointerType::X_PREV;
                    } 
                    else 
                    {
                        length_table_row[it_y] = length_table_row[it_y - 1];
                        pointer_table[it_x][it_y] = PointerType::Y_PREV;
                    }
                }
                sequence_length_x_y_prev = sequence_length_x_y_now;
            }
        }
        return std::make_pair(length_table_row[size_y], std::move(pointer_table));
    }
\end{minted}

\subsection*{15.4-5}

\begin{minted}[xleftmargin=20pt,linenos]{cpp}
    // O(n^2)
    std::vector<int> LongestIncreasingSubsequence
        (const std::vector<int>& sequence)
    {
        size_t i, j, size, sequence_length_it, max_length_index;
        size = sequence.size();
        std::vector<size_t> length(size), prev(size);
        // compute
        max_length_index = 0;
        for (i = 0; i < size; ++i)
        {
            length[i] = 1;
            for (j = 0; j < i; ++j)
            {
                if (sequence[j] < sequence[i])
                {
                    sequence_length_it = length[j] + 1;
                    if (sequence_length_it > length[i])
                    {
                        length[i] = sequence_length_it;
                        prev[i] = j;
                    }
                }
            }
            if (length[i] > length[max_length_index])
                max_length_index = i;
        }
        // output
        std::vector<int> result(length[max_length_index]);
        for (i = length[max_length_index] - 1; i < length[max_length_index]; --i)
        {
            result[i] = sequence[max_length_index];
            max_length_index = prev[max_length_index];
        }
        return result;
    }
\end{minted}

\noindent
We also can sort the sequence and find the longest common sequence 
between the sequence and the sorted sequence.

\subsection*{15.4-6}

\begin{minted}[xleftmargin=20pt,linenos]{cpp}
    // O(nlgn)
    std::vector<int> LongestIncreasingSubsequenceBinarySerach
        (const std::vector<int>& sequence)
    {
        size_t i, size, seq_index;
        int lower, upper, middle;
        size = sequence.size();
        /**
        * prev
        * index is the element index of sequence
        * value is the element index of sequence 
        *      such that the element is the prev element 
        *      in the increasing subsequence
        * ---------------------------------------------------
        * sub_seq
        * index is length of the increasing subsequence
        * value is the element index of sequence
        */
        std::vector<size_t> prev(size), sub_seq;
        sub_seq.reserve(size);
        // compute
        for (i = 0; i < size; ++i)
        {
            lower = 0;
            upper = sub_seq.size() - 1;
            while (lower <= upper)
            {
                middle = lower + ((upper - lower) >> 1);
                if (sequence[sub_seq[middle]] > sequence[i])
                {
                    upper = middle - 1;
                }
                else if (sequence[sub_seq[middle]] < sequence[i])
                {
                    lower = middle + 1;
                }
                else
                {
                    lower = middle;
                    break;
                }
            }
            // sequence[i] <= sequence[sub_seq[lower]] must be true
            if (lower >= sub_seq.size())
                sub_seq.push_back(i);
            else
                sub_seq[lower] = i;
            if (lower > 0)
                prev[i] = sub_seq[lower - 1];
        }
        // output
        size = sub_seq.size();
        std::vector<int> result(size);
        seq_index = sub_seq[size - 1];
        for (i = size - 1; i < size; --i)
        {
            result[i] = sequence[seq_index];
            seq_index = prev[seq_index];
        }
        return result;
    }
\end{minted}

\section*{15.5}

\subsection*{15.5-1}

\begin{minted}[xleftmargin=20pt,linenos]{cpp}
    void ConstructOptimalBST(std::ostream& os, const Table& root, 
        const std::string& suffix, size_t i, size_t j)
    {
        if (j == i - 1)
        {
            os << "d_" << j << " is the " << suffix << "\n";
        }
        else
        {
            size_t r = root[i][j];
            os << "k_" << r << " is the " << suffix << "\n";
            ConstructOptimalBST(os, root, 
                "left child of k_" + std::to_string(r), i, r - 1);
            ConstructOptimalBST(os, root, 
                "right child of k_" + std::to_string(r), r + 1, j);
        }
    }

    void ConstructOptimalBST(std::ostream& os, const Table& root)
    {
        ConstructOptimalBST(os, root, "root", 1, root.size() - 1);
    }
\end{minted}

\subsection*{15.5-2}

\noindent
cost: $3.12$

\noindent
structure:\\
$k_5$ is the root\\
$k_2$ is the left child of $k_5$\\
$k_1$ is the left child of $k_2$\\
$d_0$ is the left child of $k_1$\\
$d_1$ is the right child of $k_1$\\
$k_3$ is the right child of $k_2$\\
$d_2$ is the left child of $k_3$\\
$k_4$ is the right child of $k_3$\\
$d_3$ is the left child of $k_4$\\
$d_4$ is the right child of $k_4$\\
$k_7$ is the right child of $k_5$\\
$k_6$ is the left child of $k_7$\\
$d_5$ is the left child of $k_6$\\
$d_6$ is the right child of $k_6$\\
$d_7$ is the right child of $k_7$

\subsection*{15.5-3}

\noindent
The $\proc{Optimal-BST}$ would still take $Theta(n^3)$.

\noindent
According to equation $(15.12)$, 
it takes $\Theta(j-i)$ to computer $w(i,j)$ each time.
Hence the total time contributed by line $9$ in $\proc{Optimal-BST}$ is

\begin{equation*}
    \begin{split}
        \sum\limits_{l=1}^n\sum\limits_{i=1}^{n-l+1}(c\cdot(j-i)) 
        &= c\cdot\sum\limits_{l=1}^n\sum\limits_{i=1}^{n-l+1}((i+l-1)-i) \\
        &= c\cdot\sum\limits_{l=1}^n(n-(l-1))(l-1) \\
        &= c\cdot\sum\limits_{z=0}^{n-1}(n-z)z \\
        &= c\cdot\sum\limits_{z=0}^{n-1}(n-z)z \\
        &= \Theta(n^3)
    \end{split}
\end{equation*}

\subsection*{15.5-4}

\begin{minted}[xleftmargin=20pt,linenos]{cpp}
    // O(n^2)
    std::pair<Table, Table> OptimalBST(const Row& p, const Row& q)
    {
        size_t q_size, i, j, l, r;
        int t;
        q_size = q.size();// n + 1
        Table e(q_size + 1, Row(q_size)), w(q_size + 1, Row(q_size)),
            root(q_size, Row(q_size));
        for (i = 1; i <= q_size; ++i)// 1 to (n + 1)
        {
            e[i][i - 1] = q[i - 1];
            // w[i][i - 1] = q[i - 1];
        }
        // l = 1
        for (i = 1; i < q_size; ++i)// 1 to n
        {
            // w[i][i] = w[i][i - 1] + p[i] + q[i];
            w[i][i] = q[i - 1] + p[i] + q[i];
            // e[i][i] = e[i][i - 1] + e[i + 1][i] + w[i][i];
            e[i][i] = q[i - 1] + q[i] + w[i][i];
            root[i][i] = i;
        }
        // l = 2 ... n
        for (l = 2; l < q_size; ++l)// 2 to n
        {
            for (i = 1; i <= q_size - l; ++i)// 1 to (n - l + 1)
            {
                j = i + l - 1;
                e[i][j] = INT_MAX;
                // root[i][j - 1] to root[i + 1][j]
                for (r = root[i][j - 1]; r <= root[i + 1][j]; ++r)
                {
                    t = e[i][r - 1] + e[r + 1][j];
                    if (t < e[i][j])
                    {
                        e[i][j] = t;
                        root[i][j] = r;
                    }
                }
                w[i][j] = w[i][j - 1] + p[j] + q[j];
                e[i][j] += w[i][j];
            }
        }
        return std::make_pair(std::move(e), std::move(root));
    }
\end{minted}

The main principle is to 
reduce the loop range of line 10 of $\proc{Optimal-BST}$
from $i...j$ to $root[i,j-1]...root[i+1,j]$.
This says modify line 10 of $\proc{Optimal-BST}$ to

\begin{codebox}
    \zi \For $r = root[i,j-1]$ \To $root[i+1,j]$
\end{codebox}

Note the modification of case of $l=1$ is skipped here. 
We primary consider $l \geq 2$.

\begin{claim}
    $\proc{Optimal-BST}$ run in $\Theta(n^2)$ time
\end{claim}

\begin{proof}
    Denote the $\For$ loop of line 5-14 as $L_1$,
    and denote the $\For$ loop of line 6-14 as $L_2$.
    In order to show $\proc{Optimal-BST}$ run in $\Theta(n^2)$ time,
    we just need to show each iteration of $L_1$ takes $\Theta(n)$ time.

    Let $t(l,i)$ be the running time of each iteration of $L_2$ with varible $l$ and $i$.

    \begin{equation*}
        \begin{split}
            t(l,i) &= r[i+1,j] - r[i,j-1] + 1 \\
            &= r[i+1,i+l-1] - r[i,i+l-2] + 1
        \end{split}
    \end{equation*}

    Each iteration of $L_1$ takes $\sum\limits_{i=1}^{n-l+1} t(l,i)$ time.
    Notice $1 \leq root[a,b] \leq n$ for all $a,b$.
    Also, by Knuth [212], 
    $root[a+k,b-1+k] \leq root[a+h,b-1+h]$ for all $k \leq h$.
    Hence we have \mbox{$root[a+h,b-1+h] - root[a+k,b-1+k] \leq n$} for all $k \leq h$.

    \begin{equation*}
        \begin{split}
            \sum\limits_{l=1}^{n}\sum\limits_{i=1}^{n-l+1} t(l,i) 
            &= \sum\limits_{l=1}^{n}\sum\limits_{i=1}^{n-l+1} (r[i+1,i+l-1] - r[i,i+l-2] + 1) \\
            &= \sum\limits_{l=1}^{n} (r[n-l+2,n] - r[1,l-1] + (n-l+1))
        \end{split}
    \end{equation*}

    We claim $r[n-l+2,n] - r[1,l-1] \leq n$ by setting $a=1$, $b=l$, $k=0$, and $h=n-l+1$.
    Hence $\sum\limits_{l=1}^{n}\sum\limits_{i=1}^{n-l+1} t(l,i) = \Theta(n^2)$.

\end{proof}

\section*{Chapter 15 Problems}

\subsection*{15-1}

The memoized, recursive version algorithm run in $O(V+E)$.

\begin{minted}[xleftmargin=20pt,linenos]{cpp}
    // O(E)
    int RecursiveLongestSimplePath(const Graph& graph, int start, int dist, 
        Row& length, Row& next)
    {
        int curr_length;
        if (length[start] >= 0)
            return length[start];
        for (const AdjListElement& adj_element : graph[start])
        {
            curr_length = adj_element.weight + RecursiveLongestSimplePath(graph, 
                adj_element.vertex, dist, length, next);
            if (curr_length > length[start])
            {
                length[start] = curr_length;
                next[start] = adj_element.vertex;
            }
        }
        return length[start];
    }

    // O(V + E)
    std::list<int> LongestSimplePath(const Graph& graph, int start, int dist)
    {
        int vertex;
        size_t size;
        std::list<int> result;
        size = graph.size();
        Row length(size, INT_MIN), next(size);
        length[dist] = 0;
        RecursiveLongestSimplePath(graph, start, dist, length, next);
        if (length[start] < 0)
            return result;
        vertex = start;
        while (vertex != dist)
        {
            result.push_back(vertex);
            vertex = next[vertex];
        }
        result.push_back(dist);
        return result;
    }
\end{minted}

We can implement a bottom-up version algorithm 
by utilizing the topological sort.

\subsection*{15-2}

Denote $X_{ij} = \langle x_i, x_{i+1}, \cdots, x_{j-1}, x_j \rangle$
and $Z_{kt} = \langle z_k, z_{k+1}, \cdots, z_{t-1}, z_t \rangle$. 
Let $Z_{kt}$ be the LPS of $X_{ij}$. 
For all $X_{ij}$ where $n > 2$, 
we claim the following:

1. If $x_i = x_j$, 
then $z_k = z_t = x_i = x_j$
and $Z_{k+1,t-1}$ is the LPS of $X_{i+1,j-1}$.

2. If $x_i \leq x_j$,
then $z_k = z_t \neq x_i$ implies $Z_{kt}$ is the LPS of $X_{i+1,j}$.

3. If $x_i \leq x_j$,
then $z_k = z_t \neq x_j$ implies $Z_{kt}$ is the LPS of $X_{i,j-1}$.

Denote $c[i,j]$ as the length of LPS of $X_{ij}$.
For all $X_{ij}$ where $n > 2$, 
we have the following recursive solution:

$c[i,j]=
\begin{cases}
    c[i+1,j-1] + 2 & \text{if $x_i = x_j$.} \\
    \func{max}(c[i+1,j], c[i,j-1]) & \text{if $x_i \neq x_j$.}
\end{cases}$

Therefore,
we have the following bottom-up algorithm which runs in $\Theta(n^2)$:

\begin{minted}[xleftmargin=20pt,linenos]{cpp}
    // O(n^2)
    std::string LPS(const std::string& str) 
    {
        size_t size, i, j, lps_left, lps_right;
        size = str.size();
        std::string result;
        LengthTable lengths(size, LengthTableRow(size, INT_MIN));
        PointerTable pointers(size, PointerTableRow(size, NIL));
        // compute
        for (i = 0; i < size - 1; ++i)
        {
            lengths[i][i] = 1;
            j = i + 1;
            if (str[i] == str[j])
            {
                lengths[i][j] = 2;
                pointers[i][j] = COMMON;
            }
            else
            {
                lengths[i][j] = 1;
                pointers[i][j] = I_SUCC;
            }
        }
        lengths[i][i] = 1;// i == size - 1
        for (i = size - 3; i < size; --i)
        {
            for (j = i + 2; j < size; ++j)
            {
                if (str[i] == str[j])
                {
                    lengths[i][j] = lengths[i + 1][j - 1] + 2;
                    pointers[i][j] = COMMON;
                }
                else if (lengths[i + 1][j] >= lengths[i][j - 1])
                {
                    lengths[i][j] = lengths[i + 1][j];
                    pointers[i][j] = I_SUCC;
                }
                else
                {
                    lengths[i][j] = lengths[i][j - 1];
                    pointers[i][j] = J_PREV;
                }
            }
        }
        // output
        i = 0;
        j = size - 1;
        lps_left = 0;
        lps_right = lengths[i][j] - 1;
        result.resize(lengths[i][j]);
        while (i < j)
        {
            if (pointers[i][j] == I_SUCC)
            {
                ++i;
            }
            else if (pointers[i][j] == J_PREV)
            {
                --j;
            }
            else
            {
                result[lps_left] = str[i];
                result[lps_right] = str[j];
                ++i;
                --j;
                ++lps_left;
                --lps_right;
            }
        }
        if (i == j)
        {
            result[lps_left] = str[i];
        }
        return result;
    }
\end{minted}

\subsection*{15-3}

\noindent
Sort points by x-coordinate first,
and put the sorted points in a sequence
$\langle p_1, p_2, \cdots, p_n \rangle$.
Let $d(i,j)$ be the distance between $p_i$ and $p_j$.
Denote $c[i,j]$ be the shortest length of bitonic path from $p_i$ to $p_j$.
That is from $p_i$ go left to $p_1$ then go right to $p_j$.
We have the the optimal substructure 
and have the following preliminary recursive solution

\begin{equation*}
    c[i,j] = \func{min}(c[a,b] + d(a,i) + d(b,j))
\end{equation*}

\noindent
where $1 \leq a \leq i$ and $1 \leq b \leq j$.
Note we have to travel all the points 
since we do not want to skip a point,
but it is not covered by the above recursive solution,
and we want to add restrictions to $a$ and $b$.
Our goal is to determine subproblems and restrict choices of subproblems.

\noindent
Consider $i < j$ only.
$p_i$ and $p_j$ does not connect for all $i > 1$ or $j > 2$.
When $i < j - 1$, we must connect $p_{j-1}$ and $p_j$.
When $i = j - 1$, all points $p_b$ where $b < j - 1$ can connect to $p_j$.
Hence we have the following complete recursive solution:

$c[i,j]=
\begin{cases}
    d(1,2) & \text{if $i = 1$ and $j = 2$.} \\
    c[i,j-1] + d(j-1,j) & \text{if $i < j-1$.} \\
    \mathop{\func{min}}\limits_{1 \leq b < j-1}(c[b,j-1] + d(b,j)) 
        =\mathop{\func{min}}\limits_{1 \leq b < i}(c[b,i] + d(b,j))
        & \text{if $i = j - 1$.}
\end{cases}$

\noindent
Therefore, we have the following bottom-up algorithm which runs in $O(n^2)$:

\begin{minted}[xleftmargin=20pt,linenos]{cpp}
    std::vector<size_t> BitonicTour(std::vector<Point>& points)
    {
        int l_length;
        size_t size, i, j, b, result_left_p, result_right_p;
        size = points.size();
        Table t_length(size, Row(size, INT_MAX)),
            t_left_neighbour(size, Row(size));
        // r_neighbour_distance[j] == Distance(points[j - 1], points[j])
        Row r_neighbour_distance(size);
        std::vector<size_t> result(size);
        // sort points by x-coordinate
        QuickSort(points, 0, (int)size - 1);
        // compute
        for (j = 2; j < size - 1; ++j)
        {
            r_neighbour_distance[j] = Distance(points[j - 1], points[j]);
        }
        t_length[0][1] = Distance(points[0], points[1]);
        t_left_neighbour[0][1] = 0;
        for (i = 0; i < size - 1; ++i)
        {
            j = i + 1;
            for (b = 0; b < i; ++b)
            {
                l_length = t_length[b][i] + Distance(points[b], points[j]);
                if (l_length < t_length[i][j])
                {
                    t_length[i][j] = l_length;
                    t_left_neighbour[i][j] = b;
                }
            }
            for (j = i + 2; j < size - 1; ++j)
            {
                t_length[i][j] = t_length[i][j - 1] + r_neighbour_distance[j];
                t_left_neighbour[i][j] = j - 1;
            }
        }
        // output
        result_left_p = 0;
        result_right_p = size - 1;
        i = size - 2;
        j = size - 1;
        while (i != j)
        {
            if (i < j)
            {
                result[result_left_p] = j;
                ++result_left_p;
                j = t_left_neighbour[i][j];
            }
            else
            {
                result[result_right_p] = i;
                --result_right_p;
                i = t_left_neighbour[j][i];
            }
        }
        result[result_left_p] = 0;
        return result;
    }
\end{minted}

\subsection*{15-4}

\noindent
Denote $c[i, j]$ as the minimum sum of extra space (characters).
For $j < n$, we have the following preliminary recursive solution:

\begin{equation*}
    c[i,j] = \mathop{\func{min}}\limits_{i < k \leq j}
        (c[i,k-1] + c[k,j])
\end{equation*}

\noindent
Let $e(i,j) = M - j + i - \sum\limits_{k=1}^j l_k$.
Fix $j$, and let $t$ be the smallest possible integer such that
$e(t,j) \geq 0$.
For $j < n$, we have the following improved recursive solution:

\begin{equation*}
    c[1,j] = 
    \begin{cases}
        e(1,j) & \text{if $t = 1$.} \\
        \mathop{\func{min}}\limits_{t \leq k \leq j}
            (c[1,k-1] + e(k, j)) & \text{if $t \geq 2$.}
    \end{cases}
\end{equation*}

\noindent
Therefore, we have the following bottom-up algorithm which runs in $O(n^2)$:

\begin{minted}[xleftmargin=20pt,linenos]{cpp}
    std::list<int> PrintingNeatly(int line_max_char, 
        const std::vector<int>& lengths)
    {
        int size, j, k, extra;
        size = (int)(lengths.size());
        /**
        * min_sum[k + 1] is the minimum extra char
        * from the first word to the word which the length is lengths[k]
        */
        std::vector<int> min_sum(size + 1);
        std::vector<int> line_start_index(size);
        std::list<int> result;
        // compute
        min_sum[0] = 0;
        for (j = 0; j < size - 1; ++j)
        {
            min_sum[j + 1] = INT_MAX;
            extra = line_max_char + 1;
            for (k = j; k >= 0; --k)
            {
                extra = extra - 1 - lengths[k];
                if (extra < 0)
                    break;
                if (min_sum[k] + extra < min_sum[j + 1])
                {
                    min_sum[j + 1] = min_sum[k] + extra;
                    line_start_index[j] = k;
                }
            }
        }
        extra = line_max_char + 1;
        min_sum[size] = INT_MAX;
        for (k = size - 1; k >= 0; --k)
        {
            extra = extra - 1 - lengths[k];
            if (extra < 0)
                break;
            if (min_sum[k] < min_sum[size])
            {
                min_sum[size] = min_sum[k];
                line_start_index[size - 1] = k;
            }
        }
        // output
        k = size;
        while (k != 0)
        {
            k = line_start_index[k - 1];
            result.push_front(k);
        }
        return result;
    }
\end{minted}

\subsection*{15-5}

\subsubsection*{(a)}

\noindent
Denote $c[i, j]$ as edit distance from $x[1 \cdots i]$ to $y[1 \cdots j]$.
We have the following preliminary recursive solution

\begin{equation*}
    c[i,j] = \mathop{\func{min}}\limits_{1 \leq k < i \atop 1 \leq t < j}
        (c[k,t] + \alpha)
\end{equation*}

\noindent
where there is a single operation from $x[k], y[t]$ to $x[i], y[j]$
which cost $\alpha$.
After adding details, we have the following improved recursive solution:

\begin{equation*}
    c[i,j] = \func{min} 
    \begin{cases}
        c[i-2,j-2] + \text{cost(twiddle)} 
            & \text{if } x[i-1]=y[j] \text{ and } x[i]=y[j-1] \\
        c[i-1,j-1] + \text{cost(copy)} & \text{if } x[i]=y[j] \\
        c[i-1,j-1] + \text{cost(replace)} & \text{if } x[i] \neq y[j] \\
        c[i,j-1] + \text{cost(insert)} \\
        c[i-1,j] + \text{cost(delete)} \\
        c[k,j]_{0 \leq k < i} + \text{cost(kill)} 
            & \text{if } i = m \text{ and } j = n
    \end{cases}
\end{equation*}

\noindent
Therefore, we have the following bottom-up algorithm which runs in $\Theta(mn)$:

\begin{minted}[xleftmargin=20pt,linenos]{cpp}
enum Operation { COPY, REPLACE, DELETE, INSERT, TWIDDLE, KILL };

// i and j are index of edit_distance and operations,
// instead of source and target
inline void CheckCopyReplaceInsertDelete(const std::string& source,
    const std::string& target, const std::vector<int>& cost,
    std::vector< std::vector<int> >& edit_distance,
    std::vector< std::vector<Operation> >& operations,
    int i, int j)
{
    int l_cost;
    if (source[i - 1] == target[j - 1])
    {
        l_cost = edit_distance[i - 1][j - 1] + cost[COPY];
        if (l_cost < edit_distance[i][j])
        {
            edit_distance[i][j] = l_cost;
            operations[i][j] = COPY;
        }
    }
    if (source[i - 1] != target[j - 1])
    {
        l_cost = edit_distance[i - 1][j - 1] + cost[REPLACE];
        if (l_cost < edit_distance[i][j])
        {
            edit_distance[i][j] = l_cost;
            operations[i][j] = REPLACE;
        }
    }
    l_cost = edit_distance[i][j - 1] + cost[INSERT];
    if (l_cost < edit_distance[i][j])
    {
        edit_distance[i][j] = l_cost;
        operations[i][j] = INSERT;
    }
    l_cost = edit_distance[i - 1][j] + cost[DELETE];
    if (l_cost < edit_distance[i][j])
    {
        edit_distance[i][j] = l_cost;
        operations[i][j] = DELETE;
    }
}

// O(mn)
std::list<Operation> EditDistance(const std::string& source,
    const std::string& target, const std::vector<int>& cost)
{
    int m, n, i, j, kill_i_pos;
    int l_cost;
    m = (int)(source.size());
    n = (int)(target.size());
    std::list<Operation> result;
    /**
     * edit_distance[i + 1][j + 1] is the edit distance
     * from source[0 ... i] to target[0 ... j]
     */
    std::vector< std::vector<int> > edit_distance(m + 1, 
        std::vector<int>(n + 1));
    /**
     * operations[i + 1][j + 1] is the operation
     * which incremented offset of source and target to i and j
     */
    std::vector< std::vector<Operation> > operations(m + 1, 
        std::vector<Operation>(n + 1));
    // compute
    edit_distance[0][0] = 0;
    for (i = 1; i <= m; ++i)// j = 0
    {
        edit_distance[i][0] = edit_distance[i - 1][0] + cost[DELETE];
        operations[i][0] = DELETE;
    }
    for (j = 1; j <= n; ++j)// i = 0
    {
        edit_distance[0][j] = edit_distance[0][j - 1] + cost[INSERT];
        operations[0][j] = INSERT;
    }
    for (i = 1; i <= m; ++i)// j = 1
    {
        edit_distance[i][1] = INT_MAX;
        CheckCopyReplaceInsertDelete(source, target, cost, 
            edit_distance, operations, i, 1);
    }
    for (j = 2; j <= n; ++j)// i = 1
    {
        edit_distance[1][j] = INT_MAX;
        CheckCopyReplaceInsertDelete(source, target, cost, 
            edit_distance, operations, 1, j);
    }
    for (i = 2; i <= m; ++i)
    {
        for (j = 2; j <= n; ++j)
        {
            edit_distance[i][j] = INT_MAX;
            CheckCopyReplaceInsertDelete(source, target, cost, 
                edit_distance, operations, i, j);
            if (source[i - 2] == target[j - 1] && 
                source[i - 1] == target[j - 2])
            {
                l_cost = edit_distance[i - 2][j - 2] + cost[TWIDDLE];
                if (l_cost < edit_distance[i][j])
                {
                    edit_distance[i][j] = l_cost;
                    operations[i][j] = TWIDDLE;
                }
            }
        }
    }
    for (i = 0; i < m; ++i)
    {
        l_cost = edit_distance[i][n] + cost[KILL];
        if ( l_cost < edit_distance[m][n])
        {
            edit_distance[m][n] = l_cost;
            operations[m][n] = KILL;
            kill_i_pos = i;
        }
    }
    // output
    i = m;
    j = n;
    while (i != 0 || j != 0)
    {
        result.push_front(operations[i][j]);
        switch (operations[i][j])
        {
        case COPY:
            --i;
            --j;
            break;
        case REPLACE:
            --i;
            --j;
            break;
        case DELETE:
            --i;
            break;
        case INSERT:
            --j;
            break;
        case TWIDDLE:
            i -= 2;
            j -= 2;
            break;
        case KILL:
            i = kill_i_pos;
            break;
        }
    }
    return result;
}
\end{minted}

\subsubsection*{(b)}

\noindent
Disable twiddle and delete operations,
and let the remaining operations have the following costs:

cost(copy) $= -1$

cost(replace) $= +1$

cost(delete) $= +2$

cost(insert) $= +2$

\subsection*{15-6}

\noindent
Let $r(x)$ be the conviviality rating of $x$.
Denote $c[x]$ as the sum of the conviviality ratings of 
the subtree root at $x$ if $x$ will attend.
Denote $d[x]$ as the sum of the conviviality ratings of 
the subtree root at $x$ if $x$ will not attend.
We have the following recursive solution:

\begin{equation*}
    c[x] = r(x) + \sum\limits_{y \text{ is child of } x} d[y]
\end{equation*}

\begin{equation*}
    d[x] = \sum\limits_{y \text{ is child of } x} \func{max}(c[y], d[y])
\end{equation*}

\noindent
Therefore, we have the following algorithm which runs in $\Theta(n)$:

\begin{minted}[xleftmargin=20pt,linenos]{cpp}
/* declarations of function are omitted */

struct Node
{
    Node *left_child = nullptr;
    Node *right_sibling = nullptr;
    std::string name;
    int conviviality_rating;
    int dp_attend_subtree_r_sum = INT_MIN;
    int dp_not_attend_subtree_r_sum = INT_MIN;
    bool dp_attend_if_parent_not_attend;
};

void DpAuxRootAttend(Node *root)
{
    Node *child;
    root->dp_attend_subtree_r_sum = root->conviviality_rating;
    for (child = root->left_child; child; child = child->right_sibling)
    {
        if (child->dp_not_attend_subtree_r_sum < 0)
            DpAuxRootNotAttend(child);
        root->dp_attend_subtree_r_sum += child->dp_not_attend_subtree_r_sum;
    }
}

void DpAuxRootNotAttend(Node *root)
{
    Node *child;
    root->dp_not_attend_subtree_r_sum = 0;
    for (child = root->left_child; child; child = child->right_sibling)
    {
        if (child->dp_attend_subtree_r_sum < 0)
            DpAuxRootAttend(child);
        if (child->dp_not_attend_subtree_r_sum < 0)
            DpAuxRootNotAttend(child);
        if (child->dp_attend_subtree_r_sum > child->dp_not_attend_subtree_r_sum)
        {
            root->dp_not_attend_subtree_r_sum += child->dp_attend_subtree_r_sum;
            child->dp_attend_if_parent_not_attend = true;
        }
        else
        {
            root->dp_not_attend_subtree_r_sum += child->dp_not_attend_subtree_r_sum;
            child->dp_attend_if_parent_not_attend = false;
        }
    }
}

void OutputListRootAttend(Node *root, std::list<Node*>& result)
{
    Node *child;
    result.push_back(root);
    for (child = root->left_child; child; child = child->right_sibling)
    {
        OutputListRootNotAttend(child, result);
    }
}

void OutputListRootNotAttend(Node *root, std::list<Node*>& result)
{
    Node *child;
    for (child = root->left_child; child; child = child->right_sibling)
    {
        if (child->dp_attend_if_parent_not_attend)
            OutputListRootAttend(child, result);
        else
            OutputListRootNotAttend(child, result);
    }
}

std::list<Node*> PlanningCompanyParty(Node *root)
{
    std::list<Node*> result;
    DpAuxRootAttend(root);
    DpAuxRootNotAttend(root);
    if (root->dp_attend_subtree_r_sum > root->dp_not_attend_subtree_r_sum)
        OutputListRootAttend(root, result);
    else
        OutputListRootNotAttend(root, result);
    return result;
}
\end{minted}

\subsection*{15-7}

\noindent
Note that path in this book is refered "walk" by some other authors.
Path is not defaulted to simple path.
This says there might exists same vertices in a path.
For more information, refer to page 1170.

\subsubsection*{(a)}

\noindent
Denote $c[v,i] = 1$ where $0 \leq i \leq k$ 
if there exists path that has 
$\langle \sigma_1, \sigma_2, \cdots, \sigma_i \rangle$ 
as its label; 
if there does not exist such path, then $c[v,i] = 0$.
Let $S = \exists u \in V$ such that $(u,v) \in E$ and $\sigma(u,v) = \sigma_i$.
We have the following recursive solution:

\begin{equation*}
    c[v,i] = 
    \begin{cases}
        1 & \text{if } i = 0 \\
        \mathop{\func{max}}\limits_{(u,v) \in E 
            \atop \sigma(u,v) = \sigma_i}
            (c[u,i-1])
            & \text{if } 0 < i \leq k \text{ and } S \text{ is true} \\
        0 & \text{if } 0 < i \leq k \text{ and } S \text{ is false}
    \end{cases}
\end{equation*}

\noindent
In another way,
we denote $d[u,i]=1$ where $0 \leq i \leq k$
if there exists path that has 
$\langle \sigma_{i+1}, \cdots, \sigma_k \rangle$ 
as its label; 
if there does not exist such path, then $d[v,i] = 0$.
Let $H = \exists v \in V$ such that $(u,v) \in E$ and $\sigma(u,v) = \sigma_{i+1}$.
We have the following recursive solution:

\begin{equation*}
    d[u,i] = 
    \begin{cases}
        1 & \text{if } i = k \\
        \mathop{\func{max}}\limits_{(u,v) \in E 
            \atop \sigma(u,v) = \sigma_{i+1}}
            (d[v,i+1])
            & \text{if } 0 \leq i < k \text{ and } H \text{ is true} \\
        0 & \text{if } 0 \leq i < k \text{ and } H \text{ is false}
    \end{cases}
\end{equation*}

\noindent
Therefore, we have the following memoized algorithm run in $O(k|V|^2)$:

\begin{minted}[xleftmargin=20pt,linenos]{cpp}
bool ViterbiAux(Graph& graph, int v, 
    const std::vector<Sound>& sigma, 
    int sigma_index, std::list<int>& result, 
    std::vector< std::vector<bool> >& dp_checked)
{
    bool exist_path;
    if (sigma_index >= sigma.size())
        return true;
    if (dp_checked[v][sigma_index])// dp
        return false;
    dp_checked[v][sigma_index] = true;
    for (const AdjListElement& edge : graph[v].adj_list)
    {
        if (sigma[sigma_index] == edge.sound)
        {
            if (ViterbiAux(graph, edge.vertex_index, sigma, 
                sigma_index + 1, result, dp_checked))
            {
                result.push_front(edge.vertex_index);
                return true;
            }
        }
    }
    return false;
}

// index of sigma start from 0
// return empty list if NO-SUCH-PATH
std::list<int> Viterbi(Graph& graph, int v0, 
    const std::vector<int>& sigma)
{
    std::list<int> result;
    std::vector< std::vector<bool> > dp_checked(graph.size(), 
        std::vector<bool>(sigma.size(), false));
    if (ViterbiAux(graph, v0, sigma, 0, result, dp_checked))
        result.push_front(v0);
    return result;
}
\end{minted}

\subsubsection*{(b)}

\noindent
Denote $c[v,i]$ where $0 \leq i \leq k$ as the maximum probability
of the path that has 
$\langle \sigma_1, \sigma_2, \cdots, \sigma_i \rangle$ 
as its label; 
if there does not exist such path, then $c[v,i] = 0$.
Let $S = \exists u \in V$ such that $(u,v) \in E$ and $\sigma(u,v) = \sigma_i$.
We have the following recursive solution:

\begin{equation*}
    c[v,i] = 
    \begin{cases}
        1 & \text{if } i = 0 \\
        \mathop{\func{max}}\limits_{(u,v) \in E 
            \atop \sigma(u,v) = \sigma_i}
            (c[u,i-1] \cdot p(u,v))
            & \text{if } 0 < i \leq k \text{ and } S \text{ is true} \\
        0 & \text{if } 0 < i \leq k \text{ and } S \text{ is false}
    \end{cases}
\end{equation*}

\noindent
In another way,
we denote $d[u,i]$ where $0 \leq i \leq k$ as the maximum probability
of the path that has 
$\langle \sigma_{i+1}, \cdots, \sigma_k \rangle$ 
as its label; 
if there does not exist such path, then $d[v,i] = 0$.
Let $H = \exists v \in V$ such that $(u,v) \in E$ and $\sigma(u,v) = \sigma_{i+1}$.
We have the following recursive solution:

\begin{equation*}
    d[u,i] = 
    \begin{cases}
        1 & \text{if } i = k \\
        \mathop{\func{max}}\limits_{(u,v) \in E 
            \atop \sigma(u,v) = \sigma_{i+1}}
            (d[v,i+1] \cdot p(u,v))
            & \text{if } 0 \leq i < k \text{ and } H \text{ is true} \\
        0 & \text{if } 0 \leq i < k \text{ and } H \text{ is false}
    \end{cases}
\end{equation*}

\noindent
Therefore, we have the following memoized algorithm run in $O(k|V|^2)$:

\begin{minted}[xleftmargin=20pt,linenos]{cpp}
double ViterbiProbabilityAux(Graph& graph, int v, 
    const std::vector<Sound>& sigma, 
    int sigma_index, std::vector< std::vector<int> >& dp_path, 
    std::vector< std::vector<double> >& dp_probability)
{
    double l_probability;
    if (sigma_index >= sigma.size())
        return 1;
    if (dp_probability[v][sigma_index] >= 0)// dp
        return dp_probability[v][sigma_index];
    dp_probability[v][sigma_index] = 0;
    for (const AdjListElement& edge : graph[v].adj_list)
    {
        if (sigma[sigma_index] == edge.sound)
        {
            l_probability = ViterbiProbabilityAux(graph, edge.vertex_index, 
                sigma, sigma_index + 1, dp_path, dp_probability)
                * edge.probability;
            if (l_probability > dp_probability[v][sigma_index])
            {
                dp_probability[v][sigma_index] = l_probability;
                dp_path[v][sigma_index] = edge.vertex_index;
            }
        }
    }
    return dp_probability[v][sigma_index];
}

// index of sigma start from 0
// return empty list if NO-SUCH-PATH
std::list<int> ViterbiProbability(Graph& graph, int v0, 
    const std::vector<int>& sigma)
{
    int l_vertex;
    size_t vertex_size, sigma_size, i;
    std::list<int> result;
    vertex_size = graph.size();
    sigma_size = sigma.size();
    std::vector< std::vector<int> > dp_path(vertex_size, 
        std::vector<int>(sigma_size));
    std::vector< std::vector<double> > dp_probability(vertex_size, 
        std::vector<double>(sigma_size, -1));
    if (ViterbiProbabilityAux(graph, v0, sigma, 0, dp_path, 
        dp_probability) > 0)
    {
        l_vertex = v0;
        result.push_back(l_vertex);
        for (i = 0; i < sigma_size; ++i)
        {
            l_vertex = dp_path[l_vertex][i];
            result.push_back(l_vertex);
        }
    }
    return result;
}
\end{minted}

\subsection*{15-8}

\subsubsection*{(a)}

\noindent
We form the seam from the top row to the bottom row. 
Assuming that $n>1$,
there are at least two chioces of the column (same or adjacent column) 
of the next row for each pixel.
This says the number of possible seams will be doubled at least
if we increase $m$ by 1.
Thus, we obtain

\begin{equation*}
    T(r) \geq 2T(r-1)
\end{equation*}

\noindent
Hence, the number of possible seams grows in $T(m) = \Omega(2^m)$.

\subsubsection*{(b)}

\noindent
Denote $c[i,j]$ as the lowest disruption of the seam 
from $A[i,j]$ to the bottom row.
We have the following recursive solution

\begin{equation*}
    c[i,j] = 
    \begin{cases}
        \func{max}(c[i+1,j], c[i+1,j+1]) + d[i,j]
            & \text{if } 1 \leq i < m \text{ and } j = 1 \\
        \func{max}(c[i+1,j-1], c[i+1,j], c[i+1,j+1]) + d[i,j]
            & \text{if } 1 \leq i < m \text{ and } 1 < j < n \\
        \func{max}(c[i+1,j-1], c[i+1,j]) + d[i,j]
            & \text{if } 1 \leq i < m \text{ and } j = n \\
        d[i,j] & \text{if } $i = m$
    \end{cases}
\end{equation*}

\noindent
Therefore, we have the following bottom-up algorithm run in $\Theta(mn)$:

\begin{minted}[xleftmargin=20pt,linenos]{cpp}
std::vector<int> LowestDisruptionSeam(const std::vector< std::vector<int> >& disruption)
{
    int m, n, i, j;
    m = disruption.size();
    n = disruption[0].size();
    std::vector< std::vector<int> > dp_disruption_sum(m, std::vector<int>(n)), 
        dp_col_of_next_row(m - 1, std::vector<int>(n));
    std::vector<int> result;
    result.reserve(m);
    // compute
    for (j = 0; j < n; ++j)
    {
        dp_disruption_sum[m - 1][j] = disruption[m - 1][n];
    }
    for (i = m - 2; i >= 0; --i)
    {
        if (dp_disruption_sum[i + 1][0] < dp_disruption_sum[i + 1][1])
        {
            dp_disruption_sum[i][0] = dp_disruption_sum[i + 1][0] + disruption[i][0];
            dp_col_of_next_row[i][0] = 0;
        }
        else
        {
            dp_disruption_sum[i][0] = dp_disruption_sum[i + 1][1] + disruption[i][0];
            dp_col_of_next_row[i][0] = 1;
        }
        for (j = 1; j < n - 1; ++j)
        {
            if (dp_disruption_sum[i + 1][j - 1] < dp_disruption_sum[i + 1][j])
            {
                dp_disruption_sum[i][j] = dp_disruption_sum[i + 1][j - 1];
                dp_col_of_next_row[i][j] = j - 1;
            }
            else
            {
                dp_disruption_sum[i][j] = dp_disruption_sum[i + 1][j];
                dp_col_of_next_row[i][j] = j;
            }
            if (dp_disruption_sum[i + 1][j + 1] < dp_disruption_sum[i][j])
            {
                dp_disruption_sum[i][j] = dp_disruption_sum[i + 1][j + 1];
                dp_col_of_next_row[i][j] = j + 1;
            }
            dp_disruption_sum[i][j] += disruption[i][j];
        }
        if (dp_disruption_sum[i + 1][n - 2] < dp_disruption_sum[i + 1][n - 1])
        {
            dp_disruption_sum[i][n - 1] = dp_disruption_sum[i + 1][n - 2] + 
                disruption[i][n - 1];
            dp_col_of_next_row[i][n - 1] = n - 2;
        }
        else
        {
            dp_disruption_sum[i][n - 1] = dp_disruption_sum[i + 1][n - 1] + 
                disruption[i][n - 1];
            dp_col_of_next_row[i][n - 1] = n - 1;
        }
    }
    // output
    j = std::max_element(dp_disruption_sum.begin(), dp_disruption_sum.end()) - 
        dp_disruption_sum.begin();
    result.push_back(j);
    for (i = 0; i < m - 1; ++i)
    {
        j = dp_col_of_next_row[0][j];
        result.push_back(j);
    }
    return result;
}
\end{minted}

\subsection*{15-9}

\noindent
Denote $c[i,j]$ where $1 \leq i \leq j \leq n$
as the least cost to break the string $S[i..j]$
by the break points $x \in L$ such that $x \in [i,j)$.
We have the following recursive solution

\begin{equation*}
    c[i,j] = 
    \begin{cases}
        0 & \text{if } \forall x \in [i,j), x \notin L \\
        \mathop{\func{min}}\limits_{x \in L 
            \atop x \in [i,j)}
            (c[i,x] + c[x+1,j])+j-i+1
            & \text{otherwise}
    \end{cases}
\end{equation*}

\noindent
In another way, we sort the array $L[1..m]$ first,
and let $L[0] = 0$, $L[m+1] = n$.
Denote $d[i,j]$ where $0 \leq i < j \leq m+1$ 
as the least cost to break the string $S[(L[i]+1)..L[j]]$
by the break points in $L[(i+1)...(j-1)]$.
We have the following recursive solution

\begin{equation*}
    d[i,j] = 
    \begin{cases}
        0 & \text{if } j - i = 1 \\
        \mathop{\func{min}}\limits_{i < x < j}
            (d[i,x] + d[x,j])+L[j]-L[i]
            & \text{if } j - i > 1
    \end{cases}
\end{equation*}

\noindent
Therefore, we have the following bottom-up algorithm run in $\Theta(m^3)$
(process is similar to 15.2-5):

\begin{minted}[xleftmargin=20pt,linenos]{cpp}
void OutputSequence(const std::vector<int>& breaking_point, 
    const std::vector< std::vector<int> >& dp_break,
    int i, int j, std::vector<int>& result)
{
    int x;
    if (j - i == 1)
        return;
    x = dp_break[i][j];
    result.push_back(breaking_point[x]);
    OutputSequence(breaking_point, dp_break, i, x, result);
    OutputSequence(breaking_point, dp_break, x, j, result);
}

std::pair<int, std::vector<int> > BreakingString(std::vector<int>& breaking_point, int n)
{
    int m, l, i, j, x, l_cost;
    m = (int)(breaking_point.size());
    std::vector< std::vector<int> > dp_cost(m + 2, std::vector<int>(m + 2)),
        dp_break(m + 2, std::vector<int>(m + 2));
    std::vector<int> result;
    result.reserve(m);
    breaking_point.push_back(breaking_point[0]);
    breaking_point[0] = 0;
    breaking_point.push_back(n);
    Quicksort(breaking_point, 1, m);
    for (i = 0; i <= m; ++i)// l = 1
    {
        dp_cost[i][i + 1] = 0;
    }
    for (l = 2; l <= m + 2; ++l)
    {
        for (i = 0; i <= m + 1 - l; ++i)
        {
            j = i + l;
            dp_cost[i][j] = INT_MAX;
            for (x = i + 1; x < j; ++x)
            {
                l_cost = dp_cost[i][x] + dp_cost[x][j];
                if (l_cost < dp_cost[i][j])
                {
                    dp_cost[i][j] = l_cost;
                    dp_break[i][j] = x;
                }
            }
            dp_cost[i][j] += breaking_point[j] - breaking_point[i];
        }
    }
    // output
    OutputSequence(breaking_point, dp_break, 0, m + 1, result);
    return std::make_pair(dp_cost[0][m + 1], result);
}
\end{minted}

\subsection*{15-10}

\subsubsection*{(a)}

% TODO: complete the proof

\begin{claim}
    There exists an optimal investment strategy that, in each year,
    puts all the money into a single investment.
\end{claim}

\begin{proof}
    Assume there does not exists an optimal investment strategy that, 
    in each year, puts all the money into a single investment.
    This says for all optimal investment strategy,
    there exists a year such that 
    money is put in more than one inverstment.

    For each year such that 
    we shift money into $k$ inverstments where $1 < k \leq n$.
    Let $d$ be how many dollars we have at the begin of the year.
    Let $\sum\limits_{s=1}^{k} g_s = d - f_2$.
    We put money $g_1, g_2, \cdots, g_k$ 
    into $i_1, i_2, \cdots, i_k$ respectively in year $j$,
    At the end of the year $j$,
    we have $u = \sum\limits_{s=1}^{k} g_s \cdot r_{i_s j}$ dollars.
    Let $i_m$ be the investment with the highest return rate in year $j$;
    i.e. $r_{i_m j} = \func{max}(r_{i_1 j}, r_{i_2 j}, \cdots, r_{i_n j})$.
    If we put all the money into investment $i_m$,
    we have $v = \sum\limits_{s=1}^{k} g_s \cdot r_{i_m j}$ dollars 
    at the end of the year $j$.
    We observed $u \leq v$. 
    Contradition
\end{proof}

\subsubsection*{(b)}

\begin{proof}
    Suppose that the optimal investment strategy for year $10$ 
    where year $10$ put all money in $i_{10}$
    is contained in the sequence 
    $s_{10} = \langle i_1, i_2, \cdots, i_9, i_{10} \rangle$;
    i.e. year $k$ put all money in single investment $i_k$.
    Consider a year $t$ such that $1 < t < 10$.
    Let strategy 
    $s_t = \langle i_1, i_2, \cdots, i_{t-1}, i_t \rangle$.
    Assume $s_t$ is not the optimal strategy for year $t$.
    where year $t$ put all money in $i_{t}$
    Then there exists a optimal investment strategy for year $t$
    where year $t$ put all money in $i_{t}$:
    $s_t' = \langle i_1', i_2', \cdots, i_{t-1}', i_t \rangle$.
    We can utilize ``cut-and-paste'' technique to replace 
    $i_1, i_2, \cdots, i_{t-1}$ with $i_1', i_2', \cdots, i_{t-1}'$,
    and let $s_{10}' = \langle i_1', i_2', \cdots, i_{t-1}', 
    i_t \cdots, i_{10} \rangle$.
    The strategy $s_{10}'$ result a greater amount of money 
    after $10$ years than the strategy $s_{10}$.
    Contradition.
\end{proof}

\subsubsection*{(c)}

Denote $c[i,j]$ as the largest amount of money after year $j$
where year $j$ put all money in $i$.
Then we have the following recursive solution:

\begin{equation*}
    c[i,j] = 
    \begin{cases}
        10000 \cdot r_{i1} & \text{if } j = 1 \\
        \mathop{\func{max}}\limits_{1 \leq k \leq n}
        \begin{cases}
            (c[i,j-1] - f_1) \cdot r_{ij} \\
            (c[k,j-1] - f_2) \cdot r_{ij}
        \end{cases}
        & \text{if } j > 1
    \end{cases}
\end{equation*}

The following algorithm run in $\Theta(10 \cdot n^2) = \Theta(n^2)$:

\begin{minted}[xleftmargin=20pt,linenos]{cpp}
std::pair<double, std::vector<int> > InvestmentStrategy
    (const std::vector< std::vector<double> >& rate, int f_1, int f_2)
{
    int i, j, k, n;
    double l_money;
    n = (int)(rate.size());
    std::vector< std::vector<double> > dp_money(n, std::vector<double>(10)); 
    std::vector< std::vector<int> > dp_investment(n, std::vector<int>(10));
    std::vector<int> result(10);
    // compute
    for (i = 0; i < n; ++i)// j == 0
    {
        dp_money[i][0] = rate[i][0] * 10000;
    }
    for (j = 1; j < 10; ++j)
    {
        for (i = 0; i < n; ++i)
        {
            dp_money[i][j] = (dp_money[i][j - 1] - f_1) * rate[i][j];
            dp_investment[i][j] = i;
            for (k = 0; k < n; ++k)
            {
                l_money = (dp_money[k][j - 1] - f_2) * rate[i][j];
                if (l_money > dp_money[i][j])
                {
                    dp_money[i][j] = l_money;
                    dp_investment[i][j] = k;
                }
            }
        }
    }
    // output
    l_money = dp_money[0][9];
    result[9] = 0;
    for (i = 1; i < n; ++i)
    {
        if (dp_money[i][9] > l_money)
        {
            l_money = dp_money[i][9];
            result[9] = i;
        }
    }
    for (j = 9; j > 0; --j)
    {
        result[j - 1] = dp_investment[result[j]][j];
    }
    return std::make_pair(l_money, std::move(result));
}
\end{minted}

\subsubsection*{(d)}

If the restriction is imposed,
the subproblems are depend on initial amount of money.
We cannot solve the subproblems for each initial amount of money.

\end{document}
