\input{../../tex_header}

\title{Chapter 22 Solusion}
\date{12/1/2021}

\begin{document}
\maketitle

\section*{22.1}

\subsection*{22.1-1}

out-degree: $\Theta(V + E)$ by simply counting the size of each adjacency list.

\begin{minted}[xleftmargin=20pt,linenos]{cpp}
std::vector<int> OutDegree(const Graph& graph)
{
    size_t v_size, i;
    v_size = graph.adj.size();
    std::vector<int> degree(v_size);
    for (i = 0; i < v_size; ++i)
    {
        // assume graph.adj[i].size() takes O(n) 
        // where n is size of  graph.adj[i]
        degree[i] = graph.adj[i].size();
    }
    return degree;
}
\end{minted}

in-degree: $\Theta(V + E)$ by maintaining a counting table: 
each entry of the table is the counter for in-degree of the specific vertex.

\begin{minted}[xleftmargin=20pt,linenos]{cpp}
std::vector<int> InDegree(const Graph& graph)
{
    size_t v_size, i;
    v_size = graph.adj.size();
    std::vector<int> degree(v_size);
    for (i = 0; i < v_size; ++i)
    {
        for (int v : graph.adj[i])
        {
            ++degree[v];
        }
    }
    return degree;
}
\end{minted}

\subsection*{22.1-2}

Consider the following binary tree:

\Tree [.7 
[.5
    [.1 ] [.2 ]
] [.6
    [.3 ] [.4 ]
]
]

We have the following adjacency-list representation:

\begin{tikzpicture}[>=stealth] 

\matrix (M) [matrix of nodes,
   column sep=-\pgflinewidth,
   row sep=0mm,
   nodes in empty cells,
   nodes={draw, fill=gray!20,
     minimum width=.5cm, outer sep=0pt,
     minimum height=.7cm, anchor=center}, 
   column 1/.style={nodes={minimum height=.8cm}}]
{ 
  &[2mm] 5 & /  \\ 
  &[2mm] 5 & /  \\ 
  &[2mm] 6 & /  \\ 
  &[2mm] 6 & /  \\ 
  & 1 & &[2mm] 2 & &[2mm] 7 & / \\ 
  & 3 & & 4 & & 7 & / \\ 
  & 5 & & 6 & / \\
}; 

\foreach \i in {1,2,3,4,5,6,7}{ 
 \path (M-\i-1) [late options={label=left:\i}]; 
 \draw[->] (M-\i-1)--(M-\i-2.west); 
} 

\draw[->] (M-5-3.center)--(M-5-4.west); 
\draw[->] (M-5-5.center)--(M-5-6.west); 
\draw[->] (M-6-3.center)--(M-6-4.west); 
\draw[->] (M-6-5.center)--(M-6-6.west); 
\draw[->] (M-7-3.center)--(M-7-4.west); 

\end{tikzpicture}

We have the following adjacency-matrix representation:

\begin{tabular}{c c c c c c c c}
    ~ & 1 & 2 & 3 & 4 & 5 & 6 & 7 \\ \cline{2-8}
    \multicolumn{1}{c|}{1} & 0 & 0 & 0 & 0 & 1 & 0 & \multicolumn{1}{c|}{0} \\
    \multicolumn{1}{c|}{2} & 0 & 0 & 0 & 0 & 1 & 0 & \multicolumn{1}{c|}{0} \\
    \multicolumn{1}{c|}{3} & 0 & 0 & 0 & 0 & 0 & 1 & \multicolumn{1}{c|}{0} \\
    \multicolumn{1}{c|}{4} & 0 & 0 & 0 & 0 & 0 & 1 & \multicolumn{1}{c|}{0} \\
    \multicolumn{1}{c|}{5} & 1 & 1 & 0 & 0 & 0 & 0 & \multicolumn{1}{c|}{1} \\
    \multicolumn{1}{c|}{6} & 0 & 0 & 1 & 1 & 0 & 0 & \multicolumn{1}{c|}{1} \\
    \multicolumn{1}{c|}{7} & 0 & 0 & 0 & 0 & 1 & 1 & \multicolumn{1}{c|}{0} \\ \cline{2-8}
\end{tabular}

\subsection*{22.1-3}

We can compute $G^T$ from $G$ 
for the adjacency-list representation in $\Theta(V + E)$
by the following algorithm:

\begin{minted}[xleftmargin=20pt,linenos]{cpp}
AdjListGraph Transpose(const AdjListGraph& graph)
{
    size_t size, u;
    size = graph.adj.size();
    AdjListGraph target(size);
    for (u = 0; u < size; ++u)
    {
        for (int v : graph.adj[u])
        {
            target.adj[v].push_back(u);
        }
    }
    return target;
}
\end{minted}

We can compute $G^T$ from $G$ 
for the adjacency-matrix representation in $\Theta(V^2)$
by the following algorithm:
    
\begin{minted}[xleftmargin=20pt,linenos]{cpp}
AdjMatrixGraph Transpose(const AdjMatrixGraph& graph)
{
    size_t size, u, v;
    size = graph.adj.size();
    AdjMatrixGraph target(size);
    for (u = 0; u < size; ++u)
    {
        for (v = 0; v < size; ++v)
        {
            target.adj[v][u] = graph.adj[u][v];
        }
    }
    return target;
}
\end{minted}

\subsection*{22.1-4}

\begin{minted}[xleftmargin=20pt,linenos]{cpp}
AdjListGraph Equivalent(const AdjListGraph& graph)
{
    size_t size, u;
    size = graph.adj.size();
    AdjListGraph target(size);
    std::vector<bool> edge_usage;
    for (u = 0; u < size; ++u)
    {
        edge_usage = std::vector<bool>(size, false);
        edge_usage[u] = true;
        for (int v : graph.adj[u])
        {
            if (edge_usage[v] == false)
            {
                target.adj[u].push_back(v);
                edge_usage[v] = true;
            }
        }
    }
    return target;
}
\end{minted}

\subsection*{22.1-5}

We can compute $G^2$ from $G$ 
for the adjacency-list representation in $O(VE)$
by the following algorithm:

\begin{minted}[xleftmargin=20pt,linenos]{cpp}
AdjListGraph Square(const AdjListGraph& graph)
{
    size_t size, u;
    size = graph.adj.size();
    AdjListGraph result(size);
    for (u = 0; u < size; ++u)
    {
        for (int v : graph.adj[u])
        {
            result.adj[u].push_back(v);
            for (int w : graph.adj[v])
            {
                result.adj[u].push_back(w);
            }
        }
    }
    return result;
}
\end{minted}

We can compute $G^2$ from $G$ 
for the adjacency-matrix representation in $\Theta(V^3)$
by the following algorithm
(note that $G^2$ might be a not simple graph):
    
\begin{minted}[xleftmargin=20pt,linenos]{cpp}
AdjMatrixGraph Square(const AdjMatrixGraph& graph)
{
    size_t size, u, v, w;
    size = graph.Rows();
    AdjMatrixGraph result(size, size);
    for (u = 0; u < size; ++u)
    {
        for (v = 0; v < size; ++v)
        {
            if (graph[u][v])
            {
                result[u][v] = true;
                for (w = 0; w < size; ++w)
                {
                    if (graph[v][w])
                    {
                        result[u][w] = true;
                    }
                }
            }
        }
    }
    return result;
}
\end{minted}

We also can optimate computation of $G^2$ from $G$
by using Strassen algorithm.

\begin{lemma}
    Let $A=(a_{ij})$ be a $n \times n$ nonnegative matrix and $B=A^2=(b_{ij})$.
    Then $b_{uw} > 0$ if and only if 
    there exists some integer $v \in [1,n]$ such that $a_{uv} > 0$ and $a_{vw} > 0$.
\end{lemma}

\begin{proof}
    Contrapositive: 
    $b_{uw} = 0$ 
    $\Longleftrightarrow$
    $(\forall v \in \ZZ_{[1,n]}, a_{uv} = 0 \vee a_{vw} = 0)$

    According to equation (4.8) on page 75, we have

    \begin{equation*}
        b_{uw} = \sum\limits_{v=1}^{n} a_{uv} \cdot a_{vw}.
    \end{equation*}

    Note $A$ is nonnegative matrix.
    Clearly, $b_{uw} = 0$ if and only if
    $a_{uv}=0$ or $a_{vw}=0$ for all integer $v \in [1,n]$
\end{proof}

\begin{claim}
    Let $A=(a_{ij})$ be the adjacency-matrix repressentations of graph $G=(V,E)$.
    Let $B = A^2=(b_{ij})$.
    Then $b_{uw} > 0$ if and only if 
    there exists a path with exactly two edges between $u$ and $w$.
\end{claim}

\begin{proof}
    To prove the claim, we just need to show that 
    ``there exists a path with exactly two edges between $u$ and $w$''
    is equivalent to
    ``there exists some integer $v \in [1,n]$ such that $a_{uv} > 0$ and $a_{vw} > 0$''
    so we can utilize lemma 1.
    Let $v \in V$.
    We have $(u,v) \in E$ if and only if $a_{uv} > 0$.
    Similarly, $(v,w) \in E$ if and only if $a_{vw} > 0$.
    Also, $(u,v) \in E$ and $(v,w) \in E$ means
    there exists a path: $u \rightarrow v \rightarrow w$.
\end{proof}

Therefore, we have the following algorithm run in $\Theta(|V|^{\lg 7})$:

\begin{minted}[xleftmargin=20pt,linenos]{cpp}
AdjMatrixGraph SquareByStrassen(const AdjMatrixGraph& graph)
{
    size_t size, u, v, w;
    size = graph.Rows();
    AdjMatrixGraph result = StrassenMultiplication(graph, graph);
    for (u = 0; u < size; ++u)
    {
        for (v = 0; v < size; ++v)
        {
            if (graph[u][v])
            {
                result[u][v] = 1;
            }
        }
    }
    return result;
}
\end{minted}

\subsection*{22.1-6}

Notice that we can check whether a vertex is a universal sink in $\Theta(|V|)$.
However, it will take $O(|V|^2)$ to check all vertex precisely.
So, we want to constraint to a unique possible vertex and check that unique possible vertex.

\begin{claim}
    $v \in V$ is a universal sink if and only if
    $(\forall w \in V, a_{vw} = 0)$
    and 
    $(\forall u \in V \setminus \{ v \}, a_{uv} = 1)$. 
\end{claim}

Then we have

\begin{equation*}
    \begin{cases}
        a_{uv} = 1 & \text{ implies $u$ is not a universal sink,} \\
        a_{uv} = 0 \wedge u \neq v & \text{ implies $v$ is not a universal sink.}
    \end{cases}
\end{equation*}

Thus we can eliminate a candidate vertex either $u$ or $v$ in $\Theta(1)$ 
by access $a_{uv}$ if $u \neq v$.

Therefore, we have the following algorithm run in $\Theta(|V|)$:

\begin{minted}[xleftmargin=20pt,linenos]{cpp}
// graph must be a square matrix
// return vertex of universal sink
// return -1 if universal sink not exist
int UniversalSink(const Matrix& graph)
{
    size_t size, u, v;
    size = graph.size();
    // eliminate candidates
    u = 0;
    v = 1;
    while (v < size)
    {
        if (graph[u][v])
        {
            ++u;
            if (u == v)
            {
                ++v;
            }
        }
        else
        {
            ++v;
        }
    }
    // test the possible vertex u by claim 3
    for (v = 0; v < size; ++v)
    {
        if (graph[u][v])
            return -1;
    }
    for (v = 0; v < size; ++v)
    {
        if (graph[v][u] == false && u != v)
            return -1;
    }
    return u;
}
\end{minted}

The following algorithm runs in $\Theta(|V|)$ also:

\begin{minted}[xleftmargin=20pt,linenos]{cpp}
int UniversalSinkAnother(const Matrix& graph)
{
    size_t size, u, v;
    size = graph.size();
    u = 0;
    v = 0;
    while (u < size && v < size)
    {
        if (graph[u][v])
        {
            ++u;
        }
        else
        {
            ++v;
        }
    }
    if (u >= size)
        return -1;
    for (v = 0; v < size; ++v)
    {
        if (graph[u][v])
            return -1;
    }
    for (v = 0; v < size; ++v)
    {
        if (graph[v][u] == false && u != v)
            return -1;
    }
    return u;
}
\end{minted}

\subsection*{22.1-7}

Let matrix $C = B^T = (c_{ij})$.
This says $C$ is a $|E| \times |V|$ matrix,
and $c_{ij} = b_{ji}$.
Let $D = BB^T = (d_{ij})$.
Hence we have

\begin{equation*}
    d_{ij} = \sum\limits_{k \in E} b_{ik} c_{kj} = \sum\limits_{k \in E} b_{ik} b_{jk}
\end{equation*}

In conclusion, the meaning of $d_{ij}$ depends on whether $i=j$.

\textbf{Case 1}
$i = j$

$b_{ik}b_{jk} = b_{ik} = 1 = 1 \cdot 1 = -1 \cdot -1$ implies edge $k$ enters or leaves vertex $i$. 

$b_{ik}b_{jk} = b_{ik} = 0$ implies edge $k$ does not connect to vertex $i$. 

$b_{ik}b_{jk} = b_{ik} = -1$ is impossible since $b_{ik} = b_{jk}$.

Hence $d_{ij}$ means the total degree (in-degree + out-degree) of vertex $i$.

\textbf{Case 2}
$i \neq j$

$b_{ik}b_{jk} = 1 = 1 \cdot 1 = -1 \cdot -1$ is impossible 
since edge $k$ cannot enter $i$ and $j$ simultaneously,
and edge $k$ cannot leave $i$ and $j$ simultaneously.

$b_{ik}b_{jk} = 0$ implies edge $k$ does not connect to vertex $i$ and $j$. 

$b_{ik}b_{jk} = -1$ implies edge $k$ leaves vertex $i$ and enters $j$,
or edge $k$ leaves vertex $j$ and enters $i$. 

Hence $-d_{ij}$ means the number of edges connect to vertex $i$ and $j$ simultaneously.

\subsection*{22.1-8}

Expected time to determine whether an edge is in the graph: $\Theta(1)$.

Disadvantage to use hash table: 
1. we are not able to handle graphs that are not simple;
2. the worst case take $\Theta(|V|)$ time.

Suggest: 
utilize red-black trees containing keys $v$ much that $(u,v) \in E$;
add a counter (counter for unweighted graph; list for weighted graph) 
to the attributes of each node in the red-black tree to handle graphs that are not simple.

Disadvantage compared to the hash table:
expect time of red-black tree is $\Theta(\lg n)$
where $n$ is the size of elements in the red-black tree. 

\centerline{\textbf{Updating...}}

\end{document}