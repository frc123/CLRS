\input{../../tex_header}

\title{Chapter 17 Solusion}
\date{12/28/2021}

\begin{document}
\maketitle

\section*{17.1}

\subsection*{17.1-1}

No.
Consider we operate $\proc{Multpush}(S,n)$ $n$ times.
Such $n$ operations cost $\Theta(n^2)$, 
so the amortized cost is $\Theta(n)$.

Actually, we can $\proc{Multpush}$ incredible large amount of items,
so $O(1)$ of course cannot be bound on the amortized cost
of stack operations.

\subsection*{17.1-2}

Consider a $k$-bit counter where each bit in the counter is $1$.
Now, we perform $\proc{Increment}$ which flips $k+1$ bits.
Then, we perform $\proc{Decrement}$ which flips $k+1$ bits again.
Hence perform a sequence of length $n$ operations 
$\langle \proc{Increment}, \proc{Decrement}, 
\proc{Increment}, \proc{Decrement}, \cdots \rangle$
cost $\Theta(nk)$ in total.

\subsection*{17.1-3}

$n + \sum\limits_{i = 1}^{\lfloor \lg n \rfloor} (2^i - 1)
\leq n + \sum\limits_{i = 0}^{\lg n} 2^i
= n + 2^{\lg n + 1} - 1 = n + 2n - 1 = 3n - 1$

Hence the amortized cost per operation is $O(1)$.

% \section*{Chapter 17 Problems}

% \subsection*{17-1}

\centerline{\textbf{Updating...}}

\end{document}